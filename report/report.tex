\documentclass[fleqn, 11pt]{article}

\usepackage{verbatim}
\usepackage{amsmath}
\usepackage{amssymb}
\usepackage{amsthm}
\usepackage{hyperref}
\usepackage{ulem}
\usepackage{enumitem}
\usepackage[top=1.0in, left=0.75in, right=0.75in, bottom=0.75in]{geometry}
\usepackage{graphicx}

\newcommand{\myline}{
    \par
    \kern3pt % space above the rules
    \hrule height 0.5pt
    \kern2pt % space between the rules
    \hrule height 0.5pt
    \kern3pt % space below the rules
    \par
}

\usepackage[T1]{fontenc}

\newcommand{\bs}[1]{\boldsymbol{#1}}
\newcommand\norm[1]{\left\lVert#1\right\rVert}

\usepackage{array}
\usepackage{caption}
\usepackage{floatrow}
\usepackage{multirow}

\usepackage{chngcntr}
\counterwithin*{equation}{section}

\usepackage{sectsty}
\sectionfont{\centering}

\usepackage[perpage]{footmisc}

\usepackage{fancyhdr}
\pagestyle{fancy}
\fancyhf{}
\lhead{190100036 \& 190100044}
\rhead{SI424: Project}
\renewcommand{\footrulewidth}{1.0pt}
\cfoot{Page \thepage}

\setlength{\parindent}{0em}
\renewcommand{\arraystretch}{2}%

\title{SI424: Statistical Inference \\ Project Report}
\author{
    \begin{tabular}{|c|c|}
        \hline
        \textsf{Krushnakant Bhattad} & \textsf{ \hspace{5pt} Devansh Jain \hspace{5pt} } \\
        \hline
        \textsf{190100036} & \textsf{190100044}\\
        \hline
    \end{tabular}
}
\date{Autumn 2021}

\begin{document}

\maketitle
\thispagestyle{empty}
\renewcommand{\arraystretch}{1}%

\myline

\vspace{7pt}

\underline{\large {\textsc{Project Title}}}:

\medskip

%write here

\hrulefill

\vspace{10pt}

\underline{\large {\textsc{Data set used}}}:

\medskip

%write here

\hrulefill

\vspace{10pt}

\underline{\large {\textsc{Associated GitHub Repository}}}:

\medskip

The GitHub repository can accessed at:
%write here

\vspace{7pt}

\myline

\newpage

\setcounter{page}{1}

\vspace{-2em}
\myline

\vspace{10pt}

\underline{\large {\textsc{Abstract}}}:

\medskip

%write here

\vspace{7pt}

\myline

\vspace{10pt}

\underline{\large {\textsc{Common Notations}}}:

\medskip

%write here

$\mathcal{N}(\mu, \sigma^2)$ denotes the Normal distribution with mean $\mu$ and variance $\sigma^2$.

\begin{comment}
It has the probability distribution: $p(x) = \dfrac{1}{\sqrt{2 \pi \sigma^2}} \exp \left(\dfrac{-(x-\mu)^2}{2 \sigma^2}\right)$
\end{comment}

\vspace{7pt}

\myline

\newpage
\section{Parameter estimation Problem 1}
\subsection{Description}

The population growth rate of Italy follows $\mathcal{N}(\mu, \sigma^2)$ distribution. \\
We have population growth rate $g = (g_1, \cdots ,g_n)$ of different years given to us for Italy. \\
Estimated values are $\hat{\mu}_{MLE} = \mathrm{mean}(g)$ and $\hat{\sigma^2}_{MLE} = \mathrm{mean}((g - \hat{\mu}_{MLE})^2)$.

\subsection{Experiment}
From \verb!population_Country.csv!, we extract one unordered lists of population growth rate for both countries. \\
The population growth rates are computed as percentage change in population (using \verb!Total2! column) for every year. \\
The size of the unordered list is 147 for Italy. \\
We randomly choose a $K$ sized subset $(K \le 147)$ of the list. We compute the estimate of $\mu$ and $\sigma^2$ ($\hat{\mu}_{MLE}$ and $\hat{\sigma^2}_{MLE}$). We repeat this for $N$ iterations. \\
We intend to observe the variation in our estimate of $\mu$ and $\sigma^2$ for different $K$ and $N$.


\subsection{Results}


\subsection{Inference}


\newpage
\section{Parameter estimation Problem 2}
\subsection{Description}
The age of people who died in Greece in 2005 follows $\mathrm{Binomial}(p, 110)$. \\
We have age $a = (a_1, \cdots ,a_n)$ of people who died in Greece in 2005 given to us. \\
Estimated value of $p$ is $\hat{p}_{MLE} = \mathrm{mean}(a)$.

\subsection{Experiment}
\verb!mortality_Country.csv! contains total deaths per age interval, we cannot generate $a$ directly. \\
We extract percentage of deaths in Greece in 2005 for each age interval. \\
We define a procedure which returns age of a person based on this percentage distribution. \\
We do this by using the CDF generated using the obtained percentage distribution and randomly sample real value in $[0, 1]$ and use this to get the age. \\
Using the above defined procedure, we generate a $K$ sized list of ages, and compute the estimate of $p$ ($\hat{p}_{MLE}$). We repeat this for $N$ iterations. \\
We intend to observe the variation in our estimate of $p$ for different $K$ and $N$.

\subsection{Results}


\subsection{Inference}


\newpage
\section{Hypothesis Testing Problem 1}
\subsection{Description}
We have two countries - Italy and Australia. \\
We have gender ratios $r = (r_1, \cdots ,r_n)$ of different years given to us for a specific country. \\
We observe that $r_i \sim \mathrm{Uniform}[\theta-\beta, \theta+\beta]$. \\
H0: Country is Italy \\
H1: Country is Australia \\
Test1: Reject H0 if $\mathrm{mean}(r) < 1.0$ \\
Test2: Reject H0 if $\hat{\theta}_{MLE} = \dfrac{\mathrm{max}(r) + \mathrm{min}(r)}{2} < 1.0$.

\subsection{Experiment}
From \verb!population_Country.csv!, we extract one unordered list of gender ratios for both countries. \\
The gender ratios are computed as ratio of total female population (using \verb!Female2! column) and total male population (using \verb!Male2! columm) for every year. \\
The size of the unordered list is 147 for Italy and 98 for Australia. \\
We randomly choose a $K$ sized subset $(K \le 98)$ of one of the lists. We compare the output of both the tests with the true value. We repeat this for $2N$ iterations, where both countries have true value for $N$ iterations (to avoid dominance of either side of hypothesis). \\
We intend to observe the values of Type I error and Type II error for both the tests for different $K$ and $N$.

\subsection{Results}


\subsection{Inference}


\newpage
\section{Hypothesis Testing Problem 2}
\subsection{Description}
We have two countries - Italy and Australia. \\
We have population growth rate $g = (g_1, \cdots ,g_n)$ of different years given to us for a specific country. \\
We observe that, if the ``specific country'' is Italy, $g_i \sim \mathcal{N}(\mu_1, \sigma_1^2)$; and if the ``specific country'' is Australia, $g_i \sim \mathcal{N}(\mu_2, \sigma_2^2)$. \\
H0: Country is Italy \\
H1: Country is Australia \\
Test: Reject H0 if $\hat{\mu}_{MLE} = \mathrm{mean}(g) > 1.2$

\subsection{Experiment}
From \verb!population_Country.csv!, we extract one unordered lists of population growth rate for both countries. \\
The population growth rates are computed as percentage change in population (using \verb!Total2! column) for every year. \\
The size of the unordered list is 147 for Italy and 98 for Australia. \\
We randomly choose a $K$ sized subset $(K \le 98)$ of one of the lists. We compare the output of both tests with the true value. We repeat this for $2N$ iterations, where both countries have true value for $N$ iterations (to avoid dominance of either side of hypothesis). \\
We intend to observe the values of Type I error and Type II error for the test for different $K$ and $N$.

\subsection{Results}


\subsection{Inference}


\end{document}
